\documentclass[a4paper,10pt, twocolumn]{article}
\usepackage{float}
\usepackage[utf8]{inputenc}
\usepackage{enumitem}
\usepackage{titlesec}
\usepackage{hyperref}

\usepackage{graphicx}
%opening
\title{Untersuchung zum EMV-Verhalten von Mikrocontrollern}
\author{Lukas Becker}
\titleformat{\section}
  {\normalfont\scshape}{\thesection}{1em}{\center}
\begin{document}

\maketitle

\begin{section}*{}

\textbf{\emph{Zusammenfassung}}
\begin{bfseries}Im Rahmen einer etwa zweimonatigen Projektarbeit 
sollte das EMV-Verhalten 
von Mikrocontrollern  
untersucht werden. Dazu wurde zunächst ein Schaltkreis ausgewählt und dann eine 
kleine 
Messplatine entworfen. An diesem wurde dann im Labor Messungen mit 
einem Spectrum-Analyzer und einem Oszilloskop durchgeführt. Es wurden 
verschiedene Beschaltungen getestet, um die vom Schaltkreis emittierten 
Störungen zu dämpfen. Zum Schluss wurde mit 
LTSpice ein Modell für den Mikrocontroller erstellt und die Schaltung nochmal 
simuliert.
\end{bfseries}
\newline
\textbf{\emph{Schlüsselwörter}}
\begin{bfseries}
EMV, Atmel ATtiny, Störspektrum
\end{bfseries}
\end{section}
\begin{section}*{I. Einleitung}
Im Umgang mit Mikrocontrollern steht das Störverhalten dieser Bauelemente 
meist nicht im Vordergrund. Will man sich bei der Verwendung eines ICs darüber 
informieren, wie man diesen möglichst störungsfrei verwenden kann,
werden meist nur unzureichende Informationen geboten.
\newline
Das Datenblatt von Atmel zum ATtiny85 empfiehlt einen $100nF$ Kondensator 
zwischen VCC und GND, dessen Effektivität unklar ist. Meist wird nur der 
Einfluss von Störungen auf die Genauigkeit der Analog-Digital-Wandlung 
betrachtet, nicht jedoch das in externe Komponenten abgegebene Spektrum[2]. 
Schon eine erste 
Messung, bei der das Board noch über die Spannungsversorgung des USBasp 
versorgt wurde, zeigte im Spektrum des Versorgungsstroms Komponenten im Abstand 
von $2MHz$ (des verwendeten ATtiny85) 
sowie $12MHz$ (des ATmega32 auf dem USBasp) bis in den Gigahertz-Bereich.
(vgl. Abb. \ref{uart}/S. \pageref{uart})
\newline
Mikrocontroller ziehen entgegen der naiven Erwartungen und entgegen der Angaben 
im Datenblatt keinen zeitlich konstanten 
Gleichstrom.
\end{section}
\begin{section}*{II. Aufbau der Platine}
\begin{figure}[H]
  \includegraphics[width=7.7cm]{../pic/boardv2}
  \caption{Version 3 der Platine}
  \label{board}
\end{figure}
\end{section}
An erster Stelle stand ein Prototyp auf einem Experimentier-Board 
mit ordnungsgemäßer Entstörung laut Datenblatt des ATtiny85[2]. Hier konnte man 
auch schon auf einem einfachen Oszilloskop die enormen Stromimpulse in der
Versorgungsleitung gut erkennen. Um nun ein genaueres Bild des Verhaltens zu 
messen, wurde der gleiche Mikrocontroller in SO8 Bauform ausgewählt und 
auf einer kleinen Platine in einem beispielhaften Aufbau eingesetzt.
\newline 
Es wurden SMA Messbuchsen sowohl für die Messung des Stromes über 
$10\Omega$ Widerstände in den Versorgungsleitungen (bei aktueller Boardversion 
nur am GND-Pin) als auch für einen intern 
auf Masse gezogenen Output-Port angebracht. Das Top-Layer wurde als 
GND definiert, das 
Bottom-Layer als VCC, um eine reale Situation zu modellieren. Natürlich musste 
der GND-Pin des Mikrocontrollers zunächst per Leiterbahn mit dem $10\Omega$
Widerstand verbunden werden, um den Strom messen zu können. Außerdem wurde am 
digitalen Output-Port 3 eine LED mit Vorwiderstand angeschlossen, um die 
Funktionsfähigkeit sowie 
Aktivität des Mikrocontrollers zu dokumentieren. 
Als Oszillator wurde der interne LC-Oszillator des ATtiny85 verwendet, 
dessen Taktfrequenz im Makefile auf $2MHz$ spezifiziert wurde.
Die Spannungsversorgung über einen stabilisierenden Linearregler wurde zwar 
konzipiert, kam aber nicht zum Einsatz.
\begin{section}*{III. Messung}
\begin{figure}[H]
  \includegraphics[width=6cm]{../pic/block1}
  \caption{Aufbau der Messung des Stroms in der Versorgungsleitung}
  \label{block1}
\end{figure}
\begin{figure}[H]
  \includegraphics[width=6cm]{../pic/block2}
  \caption{Aufbau der Messung der Spannung an dem intern auf Masse gelegten 
Output-Port}
  \label{block2}
\end{figure}
\begin{figure}[H]
  \includegraphics[width=6cm]{../pic/block3}
  \caption{Aufbau der Messung des Stroms in der Versorgungsleitung bei 
Spannungsversorgung über die UART-Schnittstelle}
  \label{block3}
\end{figure}
Es wurde zunächst ein Oszilloskop mit einer maximalen Auflösung von 
acht Gigasamples pro Sekunde verwendet, um Daten der Stromaufnahme zur 
Modellierung in LTSpice zu erhalten. Als Spannungsversorgung wurden $4V$ mit 
einem Labornetzteil angelegt. An den $10\Omega$-Messwiderständen konnten 
steile Spannungsspitzen festgestellt werden, welche eine Höhe von $70mV$ bis 
$125mV$ aufwiesen. Diese entsprechen durch die Parallelschaltung des 
$10\Omega{}$ Widerstandes mit dem $50\Omega$ Messkopf etwa einem Strom von 
$9mA$ bis $15.6mA$. Sie traten periodisch, unabhängig von der Frequenz der 
LED, im Abstand von circa $500ns$ auf, wie in Abbildung \ref{makro} (S. 
\pageref{makro}) zu sehen ist. Die Stromimpulse sind also stark 
korreliert mit einer Taktflanke des internen 
Oszillators. 
\begin{figure}
  \includegraphics[width=7.7cm]{../pic/imp_wide}
  \caption{Oszillogramm der Messung}
  \label{makro}
\end{figure}
\newline
Die Anstiegszeit 
der Impulse betrug circa $8ns$. Ein präzises Oszillogramm der Auflösung 
$5\frac{ns}{div}$ ist in Abbildung \ref{mirko} 
(S. \pageref{mirko}) zu sehen. 
Die Messung stellte die Basis für eine Modellierung des Controllers durch eine 
Stromquelle in LTSpice dar.
\begin{figure}
  \includegraphics[width=7.7cm]{../pic/imp_cl}
  \caption{Oszillogramm eines Impulses}
  \label{mirko}
\end{figure}
\newline\newline
Bei der Messung am Spectrum-Analyzer wurde zunächst das Leistungsspektrum 
des Spannungsabfalls am Messwiderstand aufgezeichnet. Hierbei gab es mehrere 
Messaufbauten.
Zunächst wurde das Board über die $5V$ Leitung des Programmieradapters 
versorgt, 
um mögliche Effekte des ATmega32 auf dem USBasp untersuchen zu können (vgl. 
Abb. \ref{block3}/S. \pageref{block3}). Man 
stellte fest, dass das erwartete $2MHz$-periodische Spektrum des ATtiny85 
stark von einem $12MHz$-periodischen Spektrum des Quarzoszillators des USBasp 
überlagert wurde. Dieses Spektrum setzte sich bis in den GHz-Bereich fort. Bei 
etwa $620MHz$ konnte in diesem Messaufbau ($10\Omega$ Messwiderstand) 
Spektralkomponenten von $-80dBm$ nachgewiesen werden.(Abb. \ref{uart}/S. 
\pageref{uart})
\begin{figure}[H]
  \includegraphics[width=7.7cm]{../pic/spec_uart}
  \caption{Spektrum mit Spannungsversorgung über UART}
  \label{uart}
\end{figure}
Nun wurde ein rauscharmes Labornetzteil angeschlossen und das Board mit $4V$ 
versorgt (vgl. 
Abb. \ref{block1}/S. \pageref{block1}). Im Spektrum entfiel nun der 
$12MHz$-periodische Teil. Trotzdem setzte 
sich das durch die Stromimpulse erzeugte Spektrum bis weit über $100MHz$ fort, 
dies ist in Abbildung \ref{spec200mhzO100n} (S. \pageref{spec200mhzO100n})
\begin{figure}[H]
  \includegraphics[width=7.7cm]{../pic/spectrum_ohne_100n}
  \caption{Spektrum der Spannung am Messwiderstand ohne Kondensator}
  \label{spec200mhzO100n}
\end{figure}
Dann wurde überprüft, welche Einflüsse der von Atmel empfohlene Kondensator 
tatsächlich auf das Spektrum des Betriebsstroms hat. Dies ist in Abbildung 
\ref{spec200mhzM100n} (S. \pageref{spec200mhzO100n}) sowie 
Abbildung \ref{100nfadd} (S. \pageref{100nfadd})  
zu sehen. 
\begin{figure}[H]
  \includegraphics[width=7.7cm]{../pic/spectrum_mit_100n}
  \caption{Spektrum der Spannung am Messwiderstand nach Hinzufügen des 
$100nF$ Kondensators}
  \label{spec200mhzM100n}
\end{figure}
Man erkennt, dass diese Maßnahme bis zur Resonanzfrequenz des 
Kondensators eine große Wirkung zeigt, jedoch bei induktiven Verhalten des 
kapazitiven Bauteils so gut wie keinen Einfluss mehr hat[3].\newline
Zur Bestimmung der Serieninduktivität wurde nun die Resonanzfrequenz in 
folgende Formel eingesetzt[1].
\[L_s=\frac{1}{4\pi{}^2\cdot{}f^{2}_r\cdot{}C}\]
Die Berechnung der Induktivitäten ergab $L_{47nF}=8.42nH$ sowie 
$L_{100nF}=7.0nH$.
\begin{figure}
  \includegraphics[width=7.7cm]{../pic/spec100nf_add}
  \caption{Spektrum nach Hinzufügen des Kondensators}
  \label{100nfadd}
\end{figure}
\end{section}
\begin{section}*{IV. Simulation}
In der Simulation wurde ein Modell erzeugt, welches den Mikrocontroller als 
Stromquelle darstellt, die über die Bonddrahtinduktivitäten und 
Metallisierungwiderstände mit den Versorgungsleitungen verbunden ist. Die Werte 
dieser wurden geschätzt, da sie keinen erheblichen Einfluss auf das 
Spektrum haben. Der 
Stromverlauf der Quelle wurde über eine PWL-Datei eingebunden, die aus den 
Messungen am Oszilloskop konstruiert wurde (Abb. \ref{ers}/S. \pageref{ers})
\newline
\begin{figure}[H]
  \centering
  \includegraphics[width=5.0cm]{../pic/ers}
  \caption{Ersatzschaltbild des Schaltkreises}
  \label{ers}
\end{figure}
Die Serieninduktivitäten der Kondensatoren wurden den Berechnungen nach 
der Messung entnommen.
Im Folgenden wurde das Spektrum am Messwiderstand aufgezeichnet, um die 
Simulation mit der Realität abzugleichen. Es zeigte sich eine 
Resonanzfrequenz bei $6MHz$. Eine Modifikation der Kapazität des 
Abblockkondensators führte zu einer erwarteten Verschiebung der 
Resonanzfrequenz.
\end{section}
\begin{section}*{V. Verbesserungsansätze}
Um die doch recht geringe Eigenresonanzfrequenz des Koppelkondensators bei 
gleichzeitigem Erhalt der Kapazität nach oben zu verschieben, wird in  
Bastlerforen die 
Parallelschaltung des $100nF$-Kondensators mit einem 
Kondensator mit kleinerer Kapazität (z.B. $1nF$) empfohlen. Die Messungen 
konnten diese Empfehlung nicht bestätigen. Je nach Güte und 
Eigenresonanzfrequenz dieser Elemente ergaben sich zusätzliche Resonanzen, die 
in eigenen Spektralbereichen die Störkomponenten sogar verstärkten. Auch bei 
der Verwendung induktiver Komponenten (Ferritperlen, Längsdrosseln) läuft man 
Gefahr, weitere Resonanzkreise zu entwerfen[1]. Zudem liegt der Kostenpunkt 
solcher Komponenten meist über dem anderer Möglichkeiten, auch der Platzbedarf 
kann ein begrenzender Faktor sein. 
Die Bedämpfung der Versorgungsleitung mit einem Ohmschen Widerstand hat sich 
als die einfachste und wahrscheinlich effektivste Methode herauskristallisiert. 
Dieser wird verwendet, um den vom Oszillator des Mikrocontrollers erzeugten 
Strom, welcher ungehindert in die Versorgungsleitungen abfließen kann, zum Teil 
in thermische Leistung umzusetzen. 
Diese Maßnahme wurde durch Simulation des Spektrums an einem 
Leitungswiderstand verifiziert. Das Simulationsergebnis ist in Abbildung 
\ref{final}(S. \pageref{final}) zu sehen.
\begin{figure*}
  \centering
  \includegraphics[width=15cm]{../pic/comparefinal}
  \caption{Vergleich der verbesserten und der einfachen Entstörung}
  \label{final}
\end{figure*}
Die Wert des Widerstandes ist hierbei 
wählbar, jedoch abhängig davon, ob die Peak-Ströme der periodischen Impulse 
im Widerstand eine derart große Potentialdifferenz verursachen, dass es zum 
Brown-Out des Mikrocontrollers kommt. Da die gewählte Versorgungsspannung meist 
auch 
abhängig von der gewünschten Taktfrequenz ist, muss man diese Grenzen je nach 
Einsatz neu ausloten. Im Laborbeispiel betrug die Versorgungsspannung $4V$, ein 
Takt von $2MHz$ ist auch bei $2V$ Versorgungsspannung laut Datenblatt noch 
gewährleistet (Siehe Abbildung \ref{datasheet}/S. \pageref{datasheet} ). D.h. 
der maximale Spannungsabfall darf ungefähr $2V$ 
betragen. Dies entspricht bei einem Peak-Strom von $15.6mA$ einem 
Widerstand von circa $130\Omega$, wenn man die Stützung der Betriebsspannung 
durch den Abblockkondensator außer Acht lässt.
\begin{figure}[H]
  \includegraphics[width=7.7cm]{../pic/act_power_con}
  \caption{Abhängigkeit der Stromes von der Versorgungsspannung und 
Taktfrequenz}
  \label{datasheet}
\end{figure}
Ein weiteres Problem stellen Output-Ports dar, die eine niederohmige Last gegen
Masse oder VDD treiben sollen. Diese können unter Umständen 
einen Bypass zum eingebrachten Serienwiderstand in der Versorgungsleitung 
darstellen. Der Störstrom des Mikrocontrollers kann unter Umständen 
durch
Aufspannung einer großen Fläche am Endkopplungswiderstand vorbei fließen.
Wenn man das Spektrum an Port 3 (Abb. \ref{specpin3}/S. \pageref{specpin3}) mit 
dem Spektrum in der Versorgungsleitung(Abb. \ref{spec200mhzM100n}/S. 
\pageref{spec200mhzM100n}) vergleicht, stellt man fest, dass dies in der 
Realität der Fall ist.
\begin{figure}[H]
  \centering
  \includegraphics[width=7.7cm]{../pic/spectrum_pin3_100n}
  \caption{Spektrum an Pin 3 nach Hinzufügen des $100nF$ Kondensators}
  \label{specpin3}
\end{figure}
Die einfachste Lösung besteht darin, diese Output-Ports durch 
Serienwiderstände hochohmiger zu gestalten. Siehe Abbildung \ref{series} (S. 
\pageref{series}) \newline
\begin{figure}[H]
  \centering
  \includegraphics[width=3.8cm]{../pic/series}
  \caption{Bedämpfung der Ausgangsports}
  \label{series}
\end{figure}
\newpage
Durch das Anfügen eines weiteren Kondensators nach dem Widerstand wird ein 
Tiefpass entworfen, dessen Ausgangsspannung der Störspannung über der 
Versorgungsleitung und der Quelle entspricht. Dieser ist je nach Anwendung zu 
wählen. (Vgl. Abb. \ref{final}/S. \pageref{final}) 
\end{section}
\begin{section}*{VI. Fazit}

Die oben aufgeführten Maßnahmen zeigten in LTSpice bei einem 
Quellinnenwiderstand von $10\Omega$ und einem eingefügten $100\Omega$ Widerstand 
eine Verringerung des Leitungsstrom von circa $40dB$. Wie empfänglich die 
Leitung oder die Quelle für diesen Störstrom ist, lässt sich nur schwer 
postulieren, denn der Widerstand, welchen der Strom am Fuß des 
Entstörwiderstands sieht, hängt letztendlich vom Innenwiderstand der Quelle 
und Leitung bzw. deren Verhalten bei hohen Frequenzen ab.
\newline
Es ist wichtig beim Platinenentwurf darauf zu achten, den Kondensator mit 
möglichst breiten und kurzen Leitungen anzubringen, um die Induktivität der 
Leiterbahnen gering zu halten. Der Widerstand sollte vorzugsweise 
direkt auf dem Lötpad dieses Kondensators angebracht werden.\\(Vgl. Abb. 
\ref{aufbau}/S. \pageref{aufbau})\\\\
In Zukunft soll die postulierte Verbesserung real nachgemessen werden, da 
die Zeit dafür leider nicht mehr gereicht hat. Dieser Versuch stellt 
auch für Labore und Vorlesungen einen einfach zu realisierenden Einblick in das 
Thema EMV dar. Die Unterlagen stehen für Interessenten auf Anfrage zur 
Verfügung
\begin{figure}[H]
  \includegraphics[width=7.7cm]{../pic/eagle_sketch}
  \caption{Möglicher Aufbau der Entstörschaltung}
  \label{aufbau}
\end{figure}
 
\end{section}

\begin{section}*{Literaturverzeichnis}
\begin{enumerate}[label={[\arabic*]}]
  \item Prof. Hans Sapotta, "Skript Hochfrequenztechnik"
  \item Atmel,"ATtiny25/45/85 Complete"
  \item Tayo Yuden, "Multi Layer Ceramic Capacitor -datasheet-" 
\end{enumerate}
\end{section}
\begin{section}*{}
Lukas Becker studiert Elektro- und Informationstechnik an der Hochschule 
Karlsruhe. Sein Bachelorstudium wird er voraussichtlich 2018 abschließen. 
\end{section}

\end{document}
